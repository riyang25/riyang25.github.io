\documentclass[11pt]{article}
\usepackage[margin=1in]{geometry}
\pagestyle{myheadings}
\markright{Richard Yang \hfill US Physics Team FAQ \hfill}
\begin{document}
\title{US Physics Team FAQ}
\author{Richard Yang}
\date{August 2024}
\maketitle
\section*{Disclaimers} 
Everything is from my own opinion and experience. 

\noindent See Kevin Zhou’s advice if you’re looking for tips on getting started with physics olympiads. 
\section*{What I did}
\subsection*{Who are you and why did you make this?}
I’m Richard Yang, and in 2024 I qualified for the US Physics Team and was selected for EuPhO 2024. Even before that, people asked me a lot of questions about learning physics, so I thought it would be useful for me to put my answers to them all in one place.
\subsection*{What resources did you use to make camp?}
I started going through Halliday Resnick Krane in the fall of 2022, and finished in the summer. After that, I did Kevin Zhou’s handouts until April, doing a few full-length USAPhOs before the competition.
\subsection*{How long did you spend in total and per day?}
I don’t think the precise number is that important, but definitely somewhere around 500 hours in total. During the summer I could do 2-3 hours a day, but during the school year my average was closer to around an hour a day.
\subsection*{What was your previous competition experience?}
I started math competitions in 6th grade, and I’ve qualified for AIME every year since 7th. Other than that, I didn’t have any olympiad experience. I didn’t qualify for the 2023 USAPhO.
\section*{Making it}
\subsection*{How fast should I go through HRK?}
Go at a pace that (1) doesn’t significantly mess up your life besides physics and (2) doesn’t harm your learning. Plus, an awful lot of underclassmen seem to be asking me about camp nowadays. I'm not sure why you'd feel any pressure to go fast if you're a ninth grader.
\subsection*{Should I do all the problems/exercises/questions in HRK?}
Personally I mostly skipped the exercises and did around half of the problems. I think the important thing here is recognizing when your understanding is weak and you need extra work, and to adjust accordingly. If you’re not getting more than 75\% of the problems you try, I think you should back up and review.
\subsection*{Do you need the last 6 chapters of HRK?}
They’re fun and enriching to go through, but they don’t cover stuff that shows up in olympiads that often. If you’re aiming for camp though, your goal is mastery, so you should learn it at some point, whether from HRK or some other resource.
\subsection*{Can I do Morin/Purcell/Blundell/whatever without reading the chapters in HRK first?}
Try it. If it doesn’t work, go back; if it does, great!
\subsection*{Do I even need to do HRK?}
If you don’t really vibe with the book, it’s perfectly fine to use something else to cover equivalent material. It’s not like there’s some secret formula inside; physics is physics. That said, I would highly recommend you use some sort of polished resource that’s been student-vetted.
\subsection*{What’s the prerequisite for Kevin Zhou’s handouts?}
You should read the syllabus. But essentially, it’s just knowing everything in HRK.
\subsection*{Should I read the background resources for Kevin Zhou’s handouts? Are they necessary to complete as standalone books?}
I did the background resources whenever I felt like I couldn’t understand the handouts without them. You shouldn’t feel any obligation to finish a particular book. Read them when you need them!
\subsection*{Can I just learn straight from the handouts?}
Once again, the handouts assume that you’ve learnt the material at the level of HRK. There isn’t much explanation of theory; they’re focused on problem solving.
\subsection*{Could I do the mech handouts if I’ve already learned mech but not EM?}
Kevin Zhou answers this in his FAQ.
\subsection*{Do you need to do all the handouts to qualify?}
See his FAQ. But no, you don't.
\subsection*{How do you stay motivated?}
Well, do you have the right motivations in the first place? I don’t think you can stay motivated in the long term if you don’t actually like physics. Trying to get a good EC for college admissions is not a sustainable source of motivation, and tends to just make you feel worse.
That said, when you don’t feel like doing physics, that’s your brain telling you there’s something else you’d rather be doing instead. Find out what that thing is, and decide if it’s a good idea to be doing. A lot of times, it wasn’t really anything but laziness stopping me. (Importantly, burnout and sleep deprivation are not laziness.)
\subsection*{What if I’m weak in a particular topic?}
You don’t want to be stuck just hoping that it doesn’t show up on the exam. It’s better to be well rounded than to be very good at a few particular topics. 
\subsection*{What if I can’t understand a problem, even after reading its solution?}
Try asking other people! If you still don’t understand, ask another person, and so on. If you don’t know who to ask, try PHODS.
\subsection*{What’s a good “solve rate” to aim for when I’m doing olympiad problems for practice?}
In my opinion, thinking in terms of solve rate isn’t very helpful. That said, you definitely want to avoid wasting problems. I think it’s best to think of it in terms of how legible the solution is when you can’t solve it. If it’s perfectly clear what you missed and you can immediately see how to solve it, then that’s good. If you understand the solution but it takes a while, I would recommend doing more olympiad problems that test similar concepts. 
\subsection*{Approximately how much do you need to score on USAPhO to make camp?}
It depends year to year. In 2024, the cutoff was surprisingly low, below 4 problems. I think you’re better served by just trying to solve as many problems as you can, rather than fixating on some unknown cutoff.
\subsection*{Is PhysicsWOOT worth it?}
Dunno, didn’t do it. I haven’t heard very good things about it though. I’m cautious about any paid physics classes in general. I’ve spent maybe a hundred dollars on textbooks, and that’s it. You don’t need anything extra.
\subsection*{How do you balance studying for speed-contests like the F=ma and physics olympiads like the USAPhO to qualify for camp?}
A lot of campers don’t do as well on the F=ma as you’d imagine; I only scored a 20 this year. I think that if you understand mechanics at the USAPhO level, it’s just the quirks of the F=ma that would stop you from qualifying. So, it’s very important to practice on F=ma questions with similar time constraints.
\subsection*{How do you control nerves before and during contests? What is your mindset going into the exam and during the exam?}
For F=ma, I was very nervous before the competition, and I found it helpful to move around, instead of just sitting there stressing out. I did the same thing for USAPhO. My mindset for competitions is that I’m capable of solving everything on the exam, and all I need to do is execute. (This only works if you’ve studied, of course…) The only thought I have is that I need to solve, solve, solve. It’s useless to think about how everyone else is doing, worry about whether you’re doing well, or panic over not being able to solve a particular problem. Just move on, and do the best you can.
\subsection*{How do you deal with stress over competitions?}
I know it sounds rich to say “don’t take it that seriously”, but I think you’re best off if you think of physics competitions as an excuse to do physics. Anything else is just baggage.
\section*{Finances}
\subsection*{Does camp cost money?}
This year it cost something like 750 dollars.
\subsection*{If I can’t pay, can I still go?}
If you qualify and needing to pay would prevent you from participating, they tell you to contact them. I think they will be able to make it work.
\subsection*{Do they cover travel costs?}
AAPT will reimburse up to \$400 of your travel costs. You can’t use this to reduce the camp fee though.
\subsection*{Do I have to pay for meals?}
Nope! They even cover the money we spend going out for boba or ice cream or whatever.
\section*{Life}
\subsection*{Are the other campers friendly and supportive or are they competitive because they want to be on the traveling team?}
We don’t know our own ranks in the selection process, so for the vast majority of the camp we forgot that we were in a competition at all. Everyone was very friendly with each other; we’re all there to do physics, and physics is fun.
\subsection*{Is the food good?}
Quite good actually!
\subsection*{What fun things did the campers do?}
I learned how to play a lot of card games. We watched the entire 5th season of Ninjago. I played a lot of frisbee and basketball (very badly). The night before the last day, we watched movies. 
\subsection*{How much did you study outside of class?}
Maybe 3 hours in total over the 10 days, Most of the time I spent the evenings fooling around.
\subsection*{Did every camper get a separate room? Was it nice?}
Most of us were in doubles, with one or two triples. The rooms were decent enough, and you don’t really spend much time in them anyway.
\section*{Class}
\subsection*{What do you learn at camp?}
There are a number of lectures that don’t really cover much that you can’t find in Kevin Zhou’s handouts, but they’re pretty useful as a refresher. We also got some lectures on thermodynamics and fluid mechanics that were very enlightening. For labs, the coaches gave us a lot of practical tips on experimental technique. 
\subsection*{Who are the coaches?}
The coaches are awesome to be around, and they’re really happy to engage with the students. If you're curious about what life after studying physics in college might look like, you'd do very well to ask them.
\section*{Selection}
\subsection*{How is the traveling team selected?}
We’re graded on our performance in exams and experimental labs. The first exam was on the second day of classes; there were four exams in total this time. Since a lot of people come to camp with very little experimental knowledge, the first few labs are for practice only. After that, we had four graded labs this year.
\subsection*{What format are the selection exams in?}
I can’t discuss the contents of the exams themselves, but they’re multiple hours long. The topics seem to be vaguely correlated with what we covered in lectures recently before that. If you know what USAPhO or IPhO questions look like, you’re not going to get tripped up.
\subsection*{What do the labs look like?}
Basically, you’re given some setup. You’ll derive some theoretical relation between variables, and then you’ll take measurements to confirm that relation, which might look like measuring the value of g or the capacitance in a circuit.
\subsection*{Who’s eligible for making the team?}
The conditions are that you must be able to travel at the time of the competition, as well as spend around 3 hours a day studying for the next month. Sorry if you got into RSI! Otherwise, anyone can make it. 
\subsection*{Did you think you were going to make it?}
I had no idea.
\subsection*{What’s training for the IPhO like?}
In my case it was different, since in 2024 the US went to EuPhO instead. Our training was basically to do readings from Kevin Zhou's handouts, with EuPhO problems to go along with them. I imagine IPhO training is probably different. Before the competition, we spent three days at UC Santa Barbara, and it was entirely focused on experimental training.
\subsection*{How hard is it to be selected for the team?}
I don't think it's worth worrying about it. All you can do is try your best. I will note that the vast majority of people at camp aren't good at experiments, so if you have some hands-on skills, that gives you a significant edge.
\end{document}